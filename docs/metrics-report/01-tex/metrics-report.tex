\documentclass[a4paper,oneside,DIV=12,12pt]{scrartcl}

\usepackage{graphicx}

\usepackage{fontspec}
\setmainfont{PT Serif}
\setsansfont{PT Sans}
\setmonofont{PT Mono}

\usepackage{unicode-math}
\setmathfont{STIX Two Math}

\usepackage{microtype}

\usepackage{polyglossia}
\setmainlanguage{ukrainian}
\setotherlanguage{english}

\usepackage{xcolor}

\usepackage{hyperref}
\hypersetup{
	colorlinks      = false,%
	linkbordercolor = blue,%
	pdfborderstyle  = {/S/U/W 1},%
}

%%% Itemize and enumerate customization
\usepackage{enumitem,calc}
\setlist[itemize]{label=—}
%%%

%%% Document variables definitions
\newcommand{\progname}{tinyplot}

\newcommand{\theprojcode}{PLOTSCRIPT}
\newcommand{\theprojrev}{00}
\newcommand{\thedoctype}{METRICSREPORT}
\newcommand{\thedocnum}{000}
\newcommand{\thedocfullcode}{\theprojcode-\theprojrev-\thedoctype-\thedocnum}
\newcommand{\printdocfullcode}{\texttt{\thedocfullcode}}
\newcommand{\theversion}{2017-12-07-000}
%%%

%%%
\usepackage{booktabs}
\usepackage{longtable}
%%%

%%%
\usepackage{siunitx}
\sisetup{output-decimal-marker = {,}}
%%%

\renewcommand{\arraystretch}{1.2}

\begin{document}
	\begin{titlepage}
	\begin{center}
		\vspace*{\fill}
			Звіт про метрики\\
			обчислені під час перевірки якості\\
			програмного продукту для побудови графіків
			
		\vspace*{\fill}
	\end{center}
	Кодова назва проекту: \theprojcode\\
	Код документу: \printdocfullcode\\
	Версія: \theversion\\
	\end{titlepage}
	
	
	\section{Методика}
		Оскільки супровід проекту у великій мірі автоматизований, то збір деяких метрик також виконувався автоматично.
		
	\section{Метрики програмного продукту \progname}
	\label{sec:collected-metrics}
	
		Метрики, що були зібрані автоматично, наведені у табл.~\ref{tab:software-metrics}.
		\begin{longtable}[c]{lr}
				\toprule
					Метрика & Значення\\
				\midrule
				\endhead
				\bottomrule
				\caption{Метрики програмного продукту \progname }
				\endfoot
				\label{tab:software-metrics}
				
					Дефектів на рядок коду, шт.     & $0{,}01$\\
					Кількість рядків коду           & $188$\\
					Покриття коду, \%               & $70$\\
					Розмір виконуваного файлу, байт & $5314$\\
					Час виконання програми, с       & $0{,}25$\\
					Час завантаження програми, с    & $0{,}10$\\
					Цикломатична складність         & $96$ \\
		\end{longtable}
		
		Також були обчислені метрики за Халстедом~(табл.~\ref{tab:software-metrics-halstead}). Нехай $\eta_1$~— кількість унікальних операторів, $\eta_2$~— кількість унікальних операндів, $N_1$~— загальна кількість операторів та $N_2$~— загальна кількість операндів, тоді за виміряними значеннями обчислимо інші метрики за формулами. Словник програми~$\eta$:
		\[
			\eta = \eta_1 + \eta_2 = 21 + 18 = 39.
		\]
		
		Довжина програми~$N$:
		\[
			N = N_1 + N_2 = 76 + 93 = 169.
		\]
		
		Обчислена довжина програми~$\hat{N}$:
		\[
			\hat{N} = \eta_1 \log_2 \eta_1 + \eta_2 \log_2 \eta_2
			        = 21 \log_2 21 + 18 \log_2 18
					= \num{167,30}.
		\]
		
		Об'єм програми~$V$:
		\[
			V = N \cdot \log_2 \eta
			  = 169 \cdot \log_2 39
			  = \num{893,20}.
		\]
		
		Складність програми~$D$:
		\[
			D = \frac{\eta_1}{2} \cdot \frac{N_2}{\eta_2}
			  = \frac{21}{2} \cdot \frac{93}{18}
			  = \frac{217}{4}
			  = \num{54,25}.
		\]
		
		Зусилля~$E$:
		\[
			E = D \cdot V
			  = \num{54,25} \cdot \num{893,2}
			  = \num{48456,10}.
		\]
		
		\begin{longtable}[c]{llr}
				\toprule
					Метрика & Позначення & Значення\\
				\midrule
				\endhead
				\bottomrule
				\caption{Метрики програмного продукту \progname\ за Халстедом}
				\endfoot
				\label{tab:software-metrics-halstead}
				
					Кількість унікальних операторів & $\eta_1$  & $21{,}00$\\
					Кількість унікальних операндів  & $\eta_2$  & $18{,}00$\\
					Загальна кількість операторів   & $N_1$     & $76{,}00$\\
					Загальна кількість операндів    & $N_2$     & $93{,}00$\\
					Словник програми                & $\eta$    & $39{,}00$\\
					Довжина програми                & $N$       & $169{,}00$\\
					Обчислена довжина програми      & $\hat{N}$ & $167{,}30$ \\
					Об'єм                           & $V$       & $893{,}20$ \\
					Складність                      & $D$       & $54{,}25$ \\
					Зусилля                         & $E$       & $48456{,}10$ \\
		\end{longtable}
		
\end{document}