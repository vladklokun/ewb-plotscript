\documentclass[a4paper,oneside,DIV=12,12pt]{scrartcl}

\usepackage{graphicx}

\usepackage{fontspec}
\setmainfont{PT Serif}
\setsansfont{PT Sans}
\setmonofont{PT Mono}

\usepackage{unicode-math}
\setmathfont{PT Serif}

\usepackage{microtype}

\usepackage{polyglossia}
\setmainlanguage{ukrainian}
\setotherlanguage{english}

\usepackage{xcolor}

\usepackage{hyperref}
\hypersetup{
	colorlinks      = false,%
	linkbordercolor = blue,%
	pdfborderstyle  = {/S/U/W 1},%
}

\usepackage{enumitem}
\setlist[itemize]{label=—}

\newcommand{\progname}{tinyplot}

\newcommand{\theprojcode}{PLOTSCRIPT}
\newcommand{\theprojrev}{00}
\newcommand{\thedoctype}{TESTCASEDESC}
\newcommand{\thedocnum}{000}
\newcommand{\thedocfullcode}{\theprojcode-\theprojrev-\thedoctype-\thedocnum}
\newcommand{\printdocfullcode}{\texttt{\thedocfullcode}}
\newcommand{\theversion}{2017-12-07-000}

\newcommand{\classname}[1]{\texttt{#1}}
\newcommand{\funcname}[1]{\texttt{#1}}

\newcommand{\caseattrib}[1]{\noindent\textbf{#1}}

\newcommand{\printtrue}{\texttt{True}}
\newcommand{\printfalse}{\texttt{False}}

\newcommand{\filename}[1]{\texttt{#1}}

\begin{document}
	\begin{titlepage}
	\begin{center}
		\vspace*{\fill}
			Опис тестових випадків\\
			для тестування програмного продукту\\
			для побудови графіків
			
		\vspace*{\fill}
	\end{center}
	Кодова назва проекту: \theprojcode\\
	Код документу: \printdocfullcode\\
	Версія: \theversion\\
	\end{titlepage}
	
	\tableofcontents
	\newpage
	
	\section{Клас \classname{TestFloatCorrectness}}
		Клас \classname{TestFloatCorrectness} призначений для перевірки правильності роботи функції \verb|can_be_float()| з різними значеннями аргументу і містить тестові випадки у вигляді функцій, що виконують перевірки.
		
		\subsection{Позитивні тестові випадки}
		
			\subsubsection{\funcname{test\_unsigned\_int()}}
				\caseattrib{Опис:} призначена для перевірки роботи функції \verb|can_be_float()| з додатними цілими числами.
				
				\caseattrib{Очікуваний результат:} функція повертає значення \printtrue.
				
				\caseattrib{Результат тесту:} пройдений успішно.
				
			\subsubsection{\funcname{test\_signed\_int()}}
				\caseattrib{Опис:} призначена для перевірки роботи функції \verb|can_be_float()| з цілими числами зі знаком.
				
				\caseattrib{Очікуваний результат:} функція повертає значення \printtrue.
				
				\caseattrib{Результат тесту:} пройдений успішно.
				
			\subsubsection{\funcname{test\_unsigned\_real()}}
				\caseattrib{Опис:} призначена для перевірки роботи функції \verb|can_be_float()| з додатними дійсними числами.
				
				\caseattrib{Очікуваний результат:} функція повертає значення \printtrue.
				
				\caseattrib{Результат тесту:} пройдений успішно.
			
			\subsubsection{\funcname{test\_signed\_real()}}
				\caseattrib{Опис:} призначена для перевірки роботи функції \verb|can_be_float()| з дійсними числами зі знаком.
				
				\caseattrib{Очікуваний результат:} функція повертає значення \printtrue.
				
				\caseattrib{Результат тесту:} пройдений успішно.
				
			\subsubsection{\funcname{test\_exp\_notation\_positive()}}
				\caseattrib{Опис:} призначена для перевірки роботи функції \verb|can_be_float()| з додатними дійсними числами, поданими у науковій нотації.
				
				\caseattrib{Очікуваний результат:} функція повертає значення \printtrue.
				
				\caseattrib{Результат тесту:} пройдений успішно.
				
			\subsubsection{\funcname{test\_exp\_notation\_negative()}}
				\caseattrib{Опис:} призначена для перевірки роботи функції \verb|can_be_float()| з від'ємними дійсними числами, поданими у науковій нотації.
				
				\caseattrib{Очікуваний результат:} функція повертає значення \printtrue.
				
				\caseattrib{Результат тесту:} пройдений успішно.
			
		\subsection{Негативні тестові випадки}
			
			\subsubsection{\funcname{test\_string\_word()}}
				\caseattrib{Опис:} призначена для перевірки роботи функції \verb|can_be_float()| із рядком, що містить слово.
				
				\caseattrib{Очікуваний результат:} функція повертає значення \printfalse.
				
				\caseattrib{Результат тесту:} пройдений успішно.
				
			\subsubsection{\funcname{test\_string\_ascii\_sentence()}}
				\caseattrib{Опис:} призначена для перевірки роботи функції \verb|can_be_float()| із рядком, що містить речення з символів, поданих у кодуванні~ASCII.
				
				\caseattrib{Очікуваний результат:} функція повертає значення \printfalse.
				
				\caseattrib{Результат тесту:} пройдений успішно.
				
			\subsubsection{\funcname{test\_string\_unicode\_sentence()}}
				\caseattrib{Опис:} призначена для перевірки роботи функції \verb|can_be_float()| із рядком, що містить речення з символів, поданих у кодуванні~Unicode.
				
				\caseattrib{Очікуваний результат:} функція повертає значення \printfalse.
				
				\caseattrib{Результат тесту:} пройдений успішно.
				
			\subsubsection{\funcname{test\_exp\_notation\_floating()}}
				\caseattrib{Опис:} призначена для перевірки роботи функції \verb|can_be_float()| із рядком, що містить число, подане у неправильному науковому форматі.
				
				\caseattrib{Очікуваний результат:} функція повертає значення \printfalse.
				
				\caseattrib{Результат тесту:} пройдений успішно.
		
	\section{Клас \classname{TestCoordinateValidation}}
		\subsection{Позитивні тестові випадки}
			\subsubsection{\funcname{test\_unsigned\_int\_coordinates()}}
				\caseattrib{Опис:} призначена для перевірки роботи функції \verb|is_valid_coordinates()| з додатними координатами.
				
				\caseattrib{Очікуваний результат:} функція повертає значення \printtrue.
				
				\caseattrib{Результат тесту:} пройдений успішно.
				
			\subsubsection{\funcname{test\_signed\_int\_coordinates()}}
				\caseattrib{Опис:} призначена для перевірки роботи функції \verb|is_valid_coordinates()| з від'ємними координатами.
				
				\caseattrib{Очікуваний результат:} функція повертає значення \printtrue.
				
				\caseattrib{Результат тесту:} пройдений успішно.
				
			\subsubsection{\funcname{test\_unsigned\_float\_coordinates()}}
				\caseattrib{Опис:} призначена для перевірки роботи функції \verb|is_valid_coordinates()| з додатними дійсними координатами.
				
				\caseattrib{Очікуваний результат:} функція повертає значення \printtrue.
				
				\caseattrib{Результат тесту:} пройдений успішно.
				
			\subsubsection{\funcname{test\_signed\_float\_coordinates()}}
				\caseattrib{Опис:} призначена для перевірки роботи функції \verb|is_valid_coordinates()| з від'ємними дійними координатами.
				
				\caseattrib{Очікуваний результат:} функція повертає значення \printtrue.
				
				\caseattrib{Результат тесту:} пройдений успішно.
		
		\subsection{Негативні тестові випадки}
			\subsubsection{\funcname{test\_merged\_number()}}
				\caseattrib{Опис:} призначена для перевірки роботи функції \verb|is_valid_coordinates()| з координатами, які не розділені жодним символом.
				
				\caseattrib{Очікуваний результат:} функція повертає значення \printfalse.
				
				\caseattrib{Результат тесту:} пройдений успішно.
	
	\section{Клас \classname{TestStringToCoordinates}}
		
		\subsection{Позитивні тестові випадки}
			\subsubsection{\funcname{test\_unsigned\_ints()}}
				\caseattrib{Опис:} призначена для перевірки роботи функції \verb|string_to_coordinates()| з рядками, що містять координати у форматі додатних цілих чисел.
				
				\caseattrib{Очікуваний результат:} функція повертає значення \printtrue.
				
				\caseattrib{Результат тесту:} пройдений успішно.
				
			\subsubsection{\funcname{test\_signed\_ints()}}
				\caseattrib{Опис:} призначена для перевірки роботи функції \verb|string_to_coordinates()| з рядками, що містять координати у форматі цілих чисел зі знаком.
				
				\caseattrib{Очікуваний результат:} функція повертає значення \printtrue.
				
				\caseattrib{Результат тесту:} пройдений успішно.
				
			\subsubsection{\funcname{test\_unsigned\_floats()}}
				\caseattrib{Опис:} призначена для перевірки роботи функції \verb|string_to_coordinates()| з рядками, що містять координати у форматі дійсних чисел без знака.
				
				\caseattrib{Очікуваний результат:} функція повертає значення \printtrue.
				
				\caseattrib{Результат тесту:} пройдений успішно.
				
			\subsubsection{\funcname{test\_signed\_floats()}}
				\caseattrib{Опис:} призначена для перевірки роботи функції \verb|string_to_coordinates()| з рядками, що містять координати у форматі дійсних чисел зі знаком.
				
				\caseattrib{Очікуваний результат:} функція повертає значення \printtrue.
				
				\caseattrib{Результат тесту:} пройдений успішно.
				
		\subsection{Негативні тестові випадки}
			У розробці негативних тестових випадків немає сенсу, оскільки цільова функція \verb|string_to_coordinates()| працює зі значеннями координат, що вже були валідовані.
			
	\section{Клас \classname{TestBuildRawCoordinates}}
		\subsection{Позитивні тестові випадки}
			\subsubsection{\funcname{test\_file\_01()}}
				\caseattrib{Опис:} призначена для перевірки роботи функції \verb|get_raw_coordinates()| на тестовому файлі \filename{test-file-01}. Порівнює фактичний результат роботи з правильним результатом.
				
				\caseattrib{Очікуваний результат:} результат виконання функції збігається з очікуваним.
				
				\caseattrib{Результат тесту:} пройдений успішно.
			
		\subsection{Негативні тестові випадки}
			\subsubsection{\funcname{test\_file\_01\_wrong()}}
				\caseattrib{Опис:} призначена для перевірки роботи функції \verb|get_raw_coordinates()| на тестовому файлі \filename{test-file-01}. Порівнює фактичний результат роботи з передбачуваним неправильним результатом.
				
				\caseattrib{Очікуваний результат:} результат виконання функції не збігається з очікуваним.
				
				\caseattrib{Результат тесту:} пройдений успішно.
\end{document}